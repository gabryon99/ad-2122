% !TeX program = lualatex

\documentclass{article}
\usepackage[a4paper, total={5.5in, 10in}]{geometry}
\usepackage[utf8]{inputenc}
\usepackage{mathtools}
\usepackage[english]{babel}
\usepackage{amsthm}
\usepackage{amsfonts}
\usepackage{calrsfs}
\usepackage{lmodern}%get scalable font
\usepackage{titling}
\usepackage{listings}
\usepackage{fancyhdr}
\usepackage{graphicx}
\usepackage[dvipsnames]{xcolor}
\usepackage{fontspec}
\usepackage{lstfiracode}
\usepackage{stmaryrd}

% Define command for show :=
\newcommand{\defeq}{\vcentcolon=}
% Set font family
\fontfamily{qpl}\selectfont 
% Calligraphic H
\DeclareMathAlphabet{\pazocal}{OMS}{zplm}{m}{n}
\newcommand{\Hb}{\pazocal{H}}
\newcommand{\Bb}{\pazocal{B}}

\newcommand*{\QEDB}{\null\nobreak\hfill\ensuremath{\square}}


\newtheorem{theorem}{Theorem}[section]
\newtheorem{corollary}{Corollary}[theorem]
\newtheorem{lemma}[theorem]{Lemma}
\newtheorem{definition*}[theorem]{Definition}

\title{Tenth hands-on: Game Theory II\\[1ex] \large Algorithm Design (2021/2022)}
\author{Gabriele Pappalardo\\Email: g.pappalardo4@studenti.unipi.it\\Department of Computer Science}
\date{May 2022}

\pretitle{
    \begin{center}
    \LARGE
    \includegraphics[width=5cm]{/Users/gabryon/Projects/Papers/AD-21-22/assets/images/unipi.png}\\
}
\posttitle{
    \end{center}
}

\setmonofont{FiraCode}[
    Scale=0.85,
    Contextuals=Alternate % Activate the calt feature
]

\lstset{
    style=FiraCodeStyle,
    breakatwhitespace=false,         
    breaklines=true,                 
    captionpos=b,                    
    keepspaces=true,                
    showspaces=false,                
    showstringspaces=false,
    basewidth=0.6em,
    showtabs=false,                  
    tabsize=2,
    frame=single,
    basicstyle=\ttfamily, % Use \ttfamily for source code listings
    commentstyle=\color{ForestGreen}
}


\setcounter{section}{0}

\begin{document}

\maketitle

\section{Introduction}

\subsection[]{Problem 1: Crossing Streets}

Two players drive up to the same intersection at the same time. If both attempt to cross, the result is a fatal traffic accident. 
The game can be modeled by a payoff matrix where crossing successfully has a payoff of $1$, not crossing pays $0$, while an accident costs $−100$.

\begin{itemize}
    \item Build the payoff matrix 
    \item Find the Nash equilibria 
    \item Find a mixed strategy Nash equilibrium 
    \item Compute the expected payoff for one player (the game is symmetric) 
\end{itemize}

\subsection[]{Problem 2: Mixed strategy for Back Stravinsky Game}

Find the mixed strategy and expected payoff for the Back Stravinsky game.

\subsection[]{Problem 3: Children to Kindergartends}

The Municipality of your city wants to implement an algorithm for the assignment of children to kindergartens that, on the one hand, takes into account 
the desiderata of families and, on the other hand, reduces `city traffic caused by taking children to school. 
Every school has a maximum capacity limit that cannot be exceeded under any circumstances.
As a form of welfare the Municipality has established the following two rules:  

\begin{enumerate}
    \item in case of a child already attending a certain school, the sibling is granted the same school; 
    \item families with only one parent have priority for schools close to the workplace.
\end{enumerate}

Model the situation as a stable matching problem and describe the payoff functions of the players. Question: what happens to twin siblings?  

\section{Solution}

\subsection{Solution for `Crossing Streets'}

The action set is build with two events $A= \{\textit{Cross}, \textit{Stop}\}$. With $D_i, i \in [1, 2]$ we define the i-th driver that is crossing the street.
The payoff table is shown in Table \ref{tbl:payoff1}.

\begin{table}[!h]
    \label{tbl:payoff1}
    \begin{center}
        \begin{tabular}{l|l|l|}
            \cline{2-3}
            $D_1$ / $D_2$                  & Cross      & Stop \\ \hline
            \multicolumn{1}{|l|}{Cross} & -100, -100 & 1, 0 \\ \hline
            \multicolumn{1}{|l|}{Stop}  & 0,1        & 0, 0 \\ \hline
            \end{tabular}
    \end{center}
    \caption{Payoff Table of Crossing Streets}
\end{table}

\noindent As we can see from the table, we have two Nash equilibria, which are the profiles: $(\textit{Cross}, \textit{Stop})$ and $(\textit{Stop}, \textit{Cross})$.\\

\noindent To find the mixed strategies for this problem, we solve the following equations:

\begin{align}
    \begin{cases}
        \mu_{D_1}(\textit{Cross})  = \mu_{D_1}(\textit{Stop}) \\
        \mu_{D_2}(\textit{Cross})  = \mu_{D_2}(\textit{Stop})
    \end{cases} \\
    \iff
    \begin{cases}
        -100q + 1(1 - q) = 0 q + 0(1 - q) \\
        -100p + 1(1 - p) = 0 p + 0(1 - p)
    \end{cases} \\
    \iff
    \begin{cases}
        q = \frac{1}{101} \\
        p = \frac{1}{101}
    \end{cases}
\end{align}

\noindent Since the game is symmetric, the expected payoff is equal for both players:

\begin{align*}
    E \llbracket \mu_{D_1} \rrbracket = E \llbracket \mu_{D_2} \rrbracket = pq(-100) + p(1- q)1 + (1- p)q0 + (1 - p)(1 - q)0 \\
    = -100pq + p - qp = -101pq + p = p (1 - 101q) = p(1 - 101p) = 0
\end{align*}

\subsection{Solution for `Mixed Strategy for Back Stravinsky Game'}

TODO.

\subsection{Solution for `Children to Kindergartends'}

TODO.

\end{document}

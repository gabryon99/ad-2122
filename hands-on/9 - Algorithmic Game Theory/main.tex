% !TeX program = lualatex

\documentclass{article}
\usepackage[a4paper, total={5.5in, 10in}]{geometry}
\usepackage[utf8]{inputenc}
\usepackage{mathtools}
\usepackage[english]{babel}
\usepackage{amsthm}
\usepackage{amsfonts}
\usepackage{calrsfs}
\usepackage{lmodern}%get scalable font
\usepackage{titling}
\usepackage{listings}
\usepackage{fancyhdr}
\usepackage{graphicx}
\usepackage[dvipsnames]{xcolor}
\usepackage{fontspec}
\usepackage{lstfiracode}
\usepackage{multirow,array}

% Define command for show :=
\newcommand{\defeq}{\vcentcolon=}
% Set font family
\fontfamily{qpl}\selectfont 
% Calligraphic H
\DeclareMathAlphabet{\pazocal}{OMS}{zplm}{m}{n}
\newcommand{\Hb}{\pazocal{H}}
\newcommand{\Bb}{\pazocal{B}}

\newcommand*{\QEDB}{\null\nobreak\hfill\ensuremath{\square}}


\newtheorem{theorem}{Theorem}[section]
\newtheorem{corollary}{Corollary}[theorem]
\newtheorem{lemma}[theorem]{Lemma}
\newtheorem{definition*}[theorem]{Definition}

\title{Ninth hands-on: Game Theory I\\[1ex] \large Algorithm Design (2021/2022)}
\author{Gabriele Pappalardo\\Email: g.pappalardo4@studenti.unipi.it\\Department of Computer Science}
\date{April 2022}

\pretitle{
    \begin{center}
    \LARGE
    \includegraphics[width=5cm]{/Users/gabryon/Projects/Papers/AD-21-22/assets/images/unipi.png}\\
}
\posttitle{
    \end{center}
}

\setmonofont{FiraCode}[
    Scale=0.85,
    Contextuals=Alternate % Activate the calt feature
]

\lstset{
    style=FiraCodeStyle,
    breakatwhitespace=false,         
    breaklines=true,                 
    captionpos=b,                    
    keepspaces=true,                
    showspaces=false,                
    showstringspaces=false,
    basewidth=0.6em,
    showtabs=false,                  
    tabsize=2,
    frame=single,
    basicstyle=\ttfamily, % Use \ttfamily for source code listings
    commentstyle=\color{ForestGreen}
}


\setcounter{section}{0}

\begin{document}

\maketitle

\section{Introduction}

\subsection{Problem 1: Diabolik}

After Diabolik's capture, Inspector Ginko is forced to free him because the judge, scared of Eva Kant's revenge, has acquitted him with an excuse. 
Outraged by the incident, the mayor of Clerville decides to introduce a new legislation to make judges personally liable for their mistakes. 
The new legislation allows the accused to sue the judge and have him punished in case of error. Consulted on the subject, 
Ginko is perplexed and decides to ask you to provide him with a formal demonstration of the correctness/incorrectness of this law. 


\subsection{Problem 2: Investment Agency}

An investment agency wants to collect a certain amount of money for a project. Aimed at convincing all the members of a group of N people 
to contribute to the fund, it proposes the following contract: each member can freely decide either to contribute with $100$ euros or not 
to contribute (retaining money on its own wallet). Independently on this choice, after one year, the fund will be rewarded with an interest of 50\% 
and uniformly redistributed among all the $N$ members of the group. Describe the game and find the Nash equilibrium. 

\section{Solutions}

\subsection{Diabolik}

The main game's \textit{players} are the \textit{thief} (Diabolik), the \textit{judge} and a third one, the \textit{public opinion}. The thief has two strategies available,
ha can sue or not sue the judge, having so $S_{\textrm{Diabolik}} = \{\textrm{Sue}, \textrm{Not Sue}\}$. In the other hand, the judge can \textit{absolve} or \textit{condemn} the defendant, therefore, his
stategy set is the following one  $S_{\textrm{Judge}} = \{\textrm{Absolve}, \textrm{Condemn}\}$.\\

\noindent The game can be modelled according to two cases:
\begin{enumerate}
    \item Diabolik is guilty.
    \item Diabolik is \textbf{not} guilty.
\end{enumerate}

\subsection{1st Case: Diabolik is guilty}

In the case where Diabolik is guilty we have the following payoff matrix (the Table \ref{tbl:diabolik1}).
\begin{table}[h!]
    \centering
    \setlength{\extrarowheight}{2pt}
    \label{tbl:diabolik1}
    \begin{tabular}{*{4}{c|}}
      \multicolumn{2}{c}{} & \multicolumn{2}{c}{Guilty Diabolik}\\\cline{3-4}
      \multicolumn{1}{c}{} &  & Sue  & Not Sue \\\cline{2-4}
      \multirow{2}*{Judge}  & Condemn & $(-1, -1)$ & $(0, -1)$ \\\cline{2-4}
      & Absolve & $(-2, -1)$ & $(0,0)$ \\\cline{2-4}
    \end{tabular}
    \caption{Payoff table when Diabolik is guilty.}
\end{table}

\noindent In the Table \ref{tbl:diabolik1} we have to consider the \textit{public opinion} as well. If the judge acts correctly,
condemning Diabolik when he is really guilty, then the public opinion will be happy about it. If the judge is corrupted and Diabolik is guilty, then the 
opinion will be upset with the judge's decision.

\subsection{2nd Case: Diabolik is not guilty}

In the case where Diabolik is not guilty we have the following payoff matrix:
\begin{table}[h!]
    \centering
    \setlength{\extrarowheight}{2pt}
    \label{tbl:diabolik2}
    \begin{tabular}{*{4}{c|}}
      \multicolumn{2}{c}{} & \multicolumn{2}{c}{Guilty Diabolik}\\\cline{3-4}
      \multicolumn{1}{c}{} &  & Sue  & Not Sue \\\cline{2-4}
      \multirow{2}*{Judge}  & Condemn & $(-2, -1)$ & $(0, -2)$ \\\cline{2-4}
      & Absolve & $(-1, -1)$ & $(0,0)$ \\\cline{2-4}
    \end{tabular}
    \caption{Payoff table when Diabolik is not guilty.}
\end{table}

\noindent In this case we can make the opposite considerations about the public opinion. If the judge condemns Diabolik, then the public opinion
will not agree with judge's decision. Otherwise, if the judge makes the right choice the public opinion is happy.

\subsection{Investment Agency}

This game can be modelled as follows:
\begin{itemize}
    \item we have $N$ \textit{players}, which are the contributors. We will label each player with $P_i$ for $1 \le i \le N$.
    \item each player $P_i$ has his/her own \textit{set of strategies} $S_i$, where $S_i = \{\textrm{Contribute}, \textrm{Not Contribute}\}$.
    \item with the functions $u: S \to \mathbb{Z}$ and $c: S \to \mathbb{Z}$ to denote \textit{payoff} and \textit{cost} for each player, defined below:
\end{itemize}

\begin{gather*}
    u(s_i) = \frac{(k \cdot 100) \cdot 1.5}{N} \quad \textrm{where} \; k = \#\textrm{contributing players, including} \; i \\
    c(s_i) = \begin{cases}
        100 & \textrm{if} \; \textrm{player} \; P_i \; \textrm{Contibutes} \\
        0 & \textrm{otherwise}
    \end{cases}
\end{gather*}

\noindent The total incomes for a player $P_i$, denoted with $g_i$, are computed using the following equation:
\begin{equation*}
    g_i = u(s_i) - c(s_i)
\end{equation*}
We can have two specific situations. In the foremost, if the incomes $g_i \ge 0$ then the player benefits from the investments, otherwise, if the incomes
are $g_i \le 0$ the contributor loses money. The important question to ask ourselves is the following one: when should I contribute for the project?
To answer that, let us build an incoming matrix, containing all the possible situations for the game. We restrict ourselves in the case of $N=2$ players (this could be easily generalized).
With $C$ we indicate a \textit{contributing} player, instead, with $NC$ a non-contributing one.

\begin{table}[h!]
    \centering
    \setlength{\extrarowheight}{2pt}
    \begin{tabular}{*{4}{c|}}
      \multicolumn{2}{c}{} & \multicolumn{2}{c}{Contributor $2$}\\\cline{3-4}
      \multicolumn{1}{c}{} &  & $C$  & $NC$ \\\cline{2-4}
      \multirow{2}*{Contributor $1$}  & $C$ & $(50,50)$ & $(-25, 75)$ \\\cline{2-4}
      & $NC$ & $(75, -25)$ & $(0,0)$ \\\cline{2-4}
    \end{tabular}
    \caption{Incomes table for $N=2$ players.}
\end{table}

\noindent Let us analyze the table:
\begin{enumerate}
    \item if the two players decide to contribute, then they will gain $50$ euros, but they will have a gain of $150$ euros.
    \item if one of the two decides to contribute, then the non-contributing player will benefit of $75$ euros for the next year, instead the contributing one has lost $25$ euros.
    \item if none of them contribute, then they will not gain or lose nothing.
\end{enumerate}

\noindent Surprisingly enough, the only one \textit{Nash Equilibrium} is obtained when none of the players contributes.

\end{document}

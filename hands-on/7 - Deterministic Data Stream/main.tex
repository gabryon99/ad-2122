% !TeX program = lualatex

\documentclass{article}
\usepackage[a4paper, total={5.5in, 10in}]{geometry}
\usepackage[utf8]{inputenc}
\usepackage{mathtools}
\usepackage[english]{babel}
\usepackage{amsthm}
\usepackage{calrsfs}
\usepackage{lmodern} %get scalable font
\usepackage{titling}
\usepackage{listings}
\usepackage{fancyhdr}
\usepackage{graphicx}
\usepackage[dvipsnames]{xcolor}
\usepackage{fontspec}
\usepackage{lstfiracode}

% Define command for show :=
\newcommand{\defeq}{\vcentcolon=}
% Set font family
\fontfamily{qpl}\selectfont 
% Calligraphic H
\DeclareMathAlphabet{\pazocal}{OMS}{zplm}{m}{n}
\newcommand{\Hb}{\pazocal{H}}
\newcommand{\Bb}{\pazocal{B}}

\newcommand*{\QEDB}{\null\nobreak\hfill\ensuremath{\square}}


\newtheorem{theorem}{Theorem}[section]
\newtheorem{corollary}{Corollary}[theorem]
\newtheorem{lemma}[theorem]{Lemma}
\newtheorem{definition*}[theorem]{Definition}

\title{Seventh hands-on: Deterministic Data Stream \\[1ex] \large Algorithm Design (2021/2022)}
\author{Gabriele Pappalardo\\Email: g.pappalardo4@studenti.unipi.it\\Department of Computer Science}
\date{April 2022}

\pretitle{
    \begin{center}
    \LARGE
    \includegraphics[width=5cm]{/Users/gabryon/Projects/Papers/AD-21-22/assets/images/unipi.png}\\
}
\posttitle{
    \end{center}
}


\setmonofont[
    Scale=0.85,
    Contextuals=Alternate % Activate the calt feature
]{FiraCode}

\lstset{
    style=FiraCodeStyle,
    breakatwhitespace=false,         
    breaklines=true,                 
    captionpos=b,                    
    keepspaces=true,                
    showspaces=false,                
    showstringspaces=false,
    basewidth=0.6em,
    showtabs=false,                  
    tabsize=2,
    frame=single,
    basicstyle=\ttfamily, % Use \ttfamily for source code listings
    commentstyle=\color{ForestGreen}
}

\setcounter{section}{0}

\begin{document}

\maketitle

\section{Introduction}

Consider a stream of $n$ items, where items can appear more than once. The problem is to find the most frequently appearing item in the stream (where ties are broken arbitrarily if more than one item satisfies the latter). 
For any fixed integer $k \ge 1$, suppose that only $k$ items and their counters can be stored, one item per memory word; namely, 
only $O(k)$ memory words of space are allowed. 
\newline

\noindent Show that the problem cannot be solved deterministically under the following rules: any algorithm can only use these $O(k)$ memory words, and read the next item of the stream (one item at a time). 
You, the adversary, have access to all the stream, and the content of these $O(k)$ memory words: 
you cannot change these words and the past items, namely, the items already read, but you can change the future, 
namely, the next item to be read. Since any algorithm must be correct for any input, you can use any amount of streams and as many distinct items as you want.  
\newline

\noindent \textit{Hints}:
\begin{enumerate}
    \item This is ``classical'' adversarial argument based on the fact that any deterministic algorithm A using $O(k)$ memory words gives the wrong answer for a suitable stream chosen by the adversary. 
    \item The stream to choose as an adversary is taken from a candidate set sufficiently large: given $O(k)$ memory works, let $f(k)$ denote the maximum number of possible situations that algorithm A can discriminate. Create a set of $C$ candidate streams, where $|C| > f(k)$: in this way there are two streams $S1$ and $S2$ that A cannot distinguish, by the pigeon principle. 
\end{enumerate}

\section{Solution}

To begin with, let us define a new function called $\textrm{moreFreq}$ such that, given an alphabet of characters $\Sigma$ we have:
\begin{gather*}
    \textrm{moreFreq}: \Sigma^{*} \to \Sigma \\
    \textrm{moreFreq}(S) = \textrm{most frequent item} \; c \; \textrm{in the stream} \; S
\end{gather*}

\noindent Where $\Sigma^{*}$ it the \textit{Kleene} closure of the alphabet used in automata theory.\\

\noindent As suggested by the hints, our main goal is to find two streams $S_1$ and $S_2$ such that $\textrm{mostFreq}(S_1) \ne \textrm{mostFreq}(S_2)$, 
but for the algorithm $A$ we have $A(S_1) = A(S_2)$. To do so, we take an universe of elements, denoted with $U$, and its subsets $\Sigma_i \subseteq U$ such
that they have cardinality $|\Sigma_i| = \frac{|U|}{2} + 1$ and $\forall i, j, i \ne j$ they satisfy the following properties:
\begin{enumerate}
    \item $|\Sigma_i \cap \Sigma_j| \ge 1$: have at least one element in common;
    \item $|\Sigma_i \setminus \Sigma_j| \ge 1$: at least one character differ in both sets.
\end{enumerate}

\noindent How many subsets satisfy the requesting criteria? We use the binomial coefficient to discover that are:
\begin{equation}
    \binom{|U|}{\frac{|U|}{2} + 1} \simeq 2^{|U|} = \eta
\end{equation}
Note that if the cardinality of $U$ is huge then the number of sets will be exponentially large!\\

\noindent From these subsets we can build the set of candidate streams $C$ such that $|C| > f(k)$. The Equations 2 and 3 show how to build
two possible streams using $\Sigma_i$ and $\Sigma_j$ alphabets. Be aware: the streams contain each character coming from its alphabet repeated
only once (this fact will be exploited later).

\begin{align}
    S_i = s_1^{(i)}s_2^{(i)}s_3^{(i)} \dots s_{\frac{|U|}{2} + 1}^{(i)} \\
    S_j = s_1^{(j)}s_2^{(j)}s_3^{(j)} \dots s_{\frac{|U|}{2} + 1}^{(j)}
\end{align}

\noindent But, how big is the set of candidate streams? We have $\eta$ alphabets which from each one we build a string as described above, and we put it in $C$.
Therefore, the set will have a cardinality of:

\begin{equation*}
    |C| = \eta = 2^{|U|}
\end{equation*}

\noindent Since we are requiring that $|C| > f(k)$ we can derive that:

\begin{equation*}
    |C| > f(k) \iff 2^{|U|} > f(k) \iff |U| > \log{f(k)}
\end{equation*}

\noindent Therefore, $\exists S_i \in \Sigma_{i}^*, S_j \in \Sigma_{j}^{*} (S_i, S_j \in C)$ such that $S_i \ne S_j $ that those streams are indistinguishable for the algorithm $A$, due to the pigeonhole principle (since $|C| > f(k)$).\\

\noindent Now, if we pick an element $x \in \Sigma_i \setminus \Sigma_j$ and $y \in \Sigma_j \setminus \Sigma_i$, such that $x \ne y$ \footnote{Since we are requesting the properties (1) and (2) we can take such $x$ and $y$.}, and we concatenate them to 
the streams $S_i, S_j$ obtaining two new streams $S_ixy$ and $S_jxy$. Giving these two streams to $A$ we have $A(S_i xy) = A(S_j xy)$ because $S_i$ and $S_j$ are
indistinguishable for $A$. But:

\begin{equation*}
    \textrm{mostFreq}(S_i xy) \ne \textrm{mostFreq}(S_j xy)
\end{equation*}

\noindent Thus, the algorithm $A$ fails.

\end{document}

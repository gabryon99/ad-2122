% !TeX program = lualatex

\documentclass{article}
\usepackage[a4paper, total={5.5in, 10in}]{geometry}
\usepackage[utf8]{inputenc}
\usepackage{mathtools}
\usepackage[english]{babel}
\usepackage{amsthm}
\usepackage{calrsfs}
\usepackage{lmodern} %get scalable font
\usepackage{titling}
\usepackage{listings}
\usepackage{fancyhdr}
\usepackage{graphicx}
\usepackage[dvipsnames]{xcolor}
\usepackage{fontspec}
\usepackage{lstfiracode}

% Define command for show :=
\newcommand{\defeq}{\vcentcolon=}
% Set font family
\fontfamily{qpl}\selectfont 
% Calligraphic H
\DeclareMathAlphabet{\pazocal}{OMS}{zplm}{m}{n}
\newcommand{\Hb}{\pazocal{H}}
\newcommand{\Bb}{\pazocal{B}}

\newcommand*{\QEDB}{\null\nobreak\hfill\ensuremath{\square}}


\newtheorem{theorem}{Theorem}[section]
\newtheorem{corollary}{Corollary}[theorem]
\newtheorem{lemma}[theorem]{Lemma}
\newtheorem{definition*}[theorem]{Definition}

\title{Sixth hands-on: Most frequent item in a stream \\[1ex] \large Algorithm Design (2021/2022)}
\author{Gabriele Pappalardo\\Email: g.pappalardo4@studenti.unipi.it\\Department of Computer Science}
\date{March 2022}

\pretitle{
    \begin{center}
    \LARGE
    \includegraphics[width=5cm]{/Users/gabryon/Projects/Papers/AD-21-22/assets/images/unipi.png}\\
}
\posttitle{
    \end{center}
}


\setmonofont[
    Scale=0.85,
    Contextuals=Alternate % Activate the calt feature
]{FiraCode}

\lstset{
    style=FiraCodeStyle,
    breakatwhitespace=false,         
    breaklines=true,                 
    captionpos=b,                    
    keepspaces=true,                
    showspaces=false,                
    showstringspaces=false,
    basewidth=0.6em,
    showtabs=false,                  
    tabsize=2,
    frame=single,
    basicstyle=\ttfamily, % Use \ttfamily for source code listings
    commentstyle=\color{ForestGreen}
}

\setcounter{section}{0}

\begin{document}

\maketitle

\section{Introduction}

Suppose to have a stream of $n$ items, so that one of them occurs $ > n/2 $ times in the stream. 
Also, the main memory is limited to keeping just $O(1)$ items and their counters (where the knowledge of the value of n is not actually required). Show how to find deterministically the most frequent item in this scenario.
\newline

\noindent \textit{Hint}: the problem cannot be solved deterministically if the most frequent item occurs $\le n/2$ times, so the fact that the frequency is $> n/2$ should be exploited.

\section{Solution}
A \textit{streaming algorithm} is an algorithm that receives its input as \textit{a stream} of data, 
and that proceeds by making only one pass through the data.\\

\noindent Assuming we have a stream of $n$ items, it is possible to find the element with the highest frequency, knowing that it occurs $n/2$ times. The proposed solution makes use of \textit{Boyer-Moore majority vote algorithm}, that is implemented in the Listing \ref{lst:majority}.

\begin{lstlisting}[language=Python,caption="Boyer-Moore majority vote algorithm",label={lst:majority}]
def boyer_moore_majority_vote(stream):

	m = None
	count = 0
	
	for i in range(len(stream)):
		if count == 0:
			m = stream[i]
			count = 1
		elif m == stream[i]:
			count += 1
		else:
			count -= 1
	
	return m
\end{lstlisting}

\noindent The algorithm uses just two local variables: one for the element and the other for the counter. Thus, the algorithm  \textit{space complexity} is $O(1)$. The algorithm will always find the element with the higher frequency while it is repeated at least $n/2$ times. When analyzing the stream the following conditions are evaluated:
\begin{enumerate}
    \item if the counter is set to zero then we mark the element, denoted by the variable $m$, equal to i-th stream value, as the most frequent element;
    \item if the i-th stream element is equal to the most frequent one, then we increase the counter by one;
    \item otherwise, if none of the above conditions have been met, then the i-th element is different from the most frequent one. Therefore, we decrease the counter.
\end{enumerate}

\noindent If $m$ is the truly majority element, the counter will be always greater or equal than $1$, because the increments will be more than the decrements.

\end{document}

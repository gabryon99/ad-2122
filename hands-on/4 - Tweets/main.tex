% !TeX program = lualatex

\documentclass{article}
\usepackage[a4paper, total={5.5in, 10in}]{geometry}
\usepackage[utf8]{inputenc}
\usepackage{mathtools}
\usepackage[english]{babel}
\usepackage{amsthm}
\usepackage{calrsfs}
\usepackage{lmodern}%get scalable font
\usepackage{titling}
\usepackage{listings}
\usepackage{fancyhdr}
\usepackage{graphicx}
\usepackage[dvipsnames]{xcolor}
\usepackage{fontspec}
\usepackage{lstfiracode}

% Define command for show :=
\newcommand{\defeq}{\vcentcolon=}
% Set font family
\fontfamily{qpl}\selectfont 
% Calligraphic H
\DeclareMathAlphabet{\pazocal}{OMS}{zplm}{m}{n}
\newcommand{\Hb}{\pazocal{H}}
\newcommand{\Bb}{\pazocal{B}}

\newcommand*{\QEDB}{\null\nobreak\hfill\ensuremath{\square}}


\newtheorem{theorem}{Theorem}[section]
\newtheorem{corollary}{Corollary}[theorem]
\newtheorem{lemma}[theorem]{Lemma}
\newtheorem{definition*}[theorem]{Definition}

\title{Fourth hands-on: Tweets \\[1ex] \large Algorithm Design (2021/2022)}
\author{Gabriele Pappalardo\\Email: g.pappalardo4@studenti.unipi.it\\Department of Computer Science}
\date{March 2022}

\pretitle{
    \begin{center}
    \LARGE
    \includegraphics[width=5cm]{/Users/gabryon/Projects/Papers/AD-21-22/assets/images/unipi.png}\\
}
\posttitle{
    \end{center}
}

\setmonofont{FiraCode}[
    Scale=0.85,
    Contextuals=Alternate % Activate the calt feature
]

\lstset{
    style=FiraCodeStyle,
    breakatwhitespace=false,         
    breaklines=true,                 
    captionpos=b,                    
    keepspaces=true,                
    showspaces=false,                
    showstringspaces=false,
    basewidth=0.6em,
    showtabs=false,                  
    tabsize=2,
    frame=single,
    basicstyle=\ttfamily, % Use \ttfamily for source code listings
    commentstyle=\color{ForestGreen}
}

\setcounter{section}{0}

\begin{document}

\maketitle

\section{Introduction}

Given a stream of Tweets collected using Twitter, answer to these questions:
\begin{itemize}
    \item Count the percentage of happy users in the different moments of the day (morning, afternoon, evening, night). Discuss what you find if compute also the percentage of unhappy users. Do the two percentages sum to 100\%? Why?
    
    \item Spell the 30 favorite words of happy users.
    
    \item Find the number of distinct words used by happy users. How could we exclude words repeated only once?

    \item Decide if, in general, happy messages are longer or shorter than unhappy messages.    
\end{itemize}

\section{Solution}

\subsection{First Problem}

The solution for the first point uses one linear counter for each moment of the day. Thus, we are going to have a total of four counters for the happy users, and other four for the unhappy ones. If we sum the obtained percentages we could obtain
a 100\% under certain conditions, that is, a user should post only a tweet. Otherwise, we could never obtain the 100\% since the mood
of a user could change during the day (see the user \textit{@missbrandii}, he/she has three negative tweets and a positive one inside the \verb+sample.csv+ dataset.). \newline

\subsection{Second Problem}

\noindent In order to spell the 30 favorite words of happy users we can use the \textit{space saving algorithm}, seen during the lectures, and query the table at the
end for the first 30 words. \newline

\subsection{Third Problem}

\noindent To find the distinct words used by happy users, we thought to use a LogLog counter combined with a Bloom Filter.
The LogLog counter is suitable to count distinct elements. Thus, to understand if a word is repeated only once we use a Bloom Filter in the following way:
\begin{enumerate}
\item Set the bit of the hashed value to 1;
\item If the bit was set to 1, then it means we already saw that word, and therefore we count it using the LogLog counter.
\end{enumerate}
A pseudo Python code solution is shown Listing \ref{lst:third}.

\begin{lstlisting}[language=Python,caption="Pseudo-Code for the third point",label={lst:third}]
def count_distinct_words(s: Stream) -> Int:
    
    counter = HyperLogLogCounter(...)
    filter = BloomFilter(...)

    word = s.next()

    while word is not None:
        if not filter.is_set(word):
            filter.set(word)
        else:
            counter.add(word)
        word = s.next()
    
    return counter.count()

\end{lstlisting}

\subsection{Fourth Problem}

\noindent The fourth point can be solved using several approaches, we used the \textit{median} as indicator. In the \verb+mini.csv+ and \verb+sample.csv+ datasets, the negatives messages are longer than the positives ones.
Instead, using the \textit{mean} as indicator, in the \verb+mini.csv+ dataset, positive messages are longer.

\end{document}

% !TeX program = lualatex

\documentclass{article}
\usepackage[a4paper, total={5.5in, 10in}]{geometry}
\usepackage[utf8]{inputenc}
\usepackage{mathtools}
\usepackage[english]{babel}
\usepackage{amsthm}
\usepackage{calrsfs}
\usepackage{lmodern} %get scalable font
\usepackage{titling}
\usepackage{listings}
\usepackage{fancyhdr}
\usepackage{graphicx}
\usepackage[dvipsnames]{xcolor}
\usepackage{fontspec}
\usepackage{lstfiracode}

% Define command for show :=
\newcommand{\defeq}{\vcentcolon=}
% Set font family
\fontfamily{qpl}\selectfont 
% Calligraphic H
\DeclareMathAlphabet{\pazocal}{OMS}{zplm}{m}{n}
\newcommand{\Hb}{\pazocal{H}}
\newcommand{\Bb}{\pazocal{B}}

\newcommand*{\QEDB}{\null\nobreak\hfill\ensuremath{\square}}


\newtheorem{theorem}{Theorem}[section]
\newtheorem{corollary}{Corollary}[theorem]
\newtheorem{lemma}[theorem]{Lemma}
\newtheorem{definition*}[theorem]{Definition}

\title{Fifth hands-on: Bloom Filters \\[1ex] \large Algorithm Design (2021/2022)}
\author{Gabriele Pappalardo\\Email: g.pappalardo4@studenti.unipi.it\\Department of Computer Science}
\date{March 2022}

\pretitle{
    \begin{center}
    \LARGE
    \includegraphics[width=5cm]{/Users/gabryon/Projects/Papers/AD-21-22/assets/images/unipi.png}\\
}
\posttitle{
    \end{center}
}


\setmonofont[
    Scale=0.85,
    Contextuals=Alternate % Activate the calt feature
]{FiraCode}

\lstset{
    style=FiraCodeStyle,
    breakatwhitespace=false,         
    breaklines=true,                 
    captionpos=b,                    
    keepspaces=true,                
    showspaces=false,                
    showstringspaces=false,
    basewidth=0.6em,
    showtabs=false,                  
    tabsize=2,
    frame=single,
    basicstyle=\ttfamily, % Use \ttfamily for source code listings
    commentstyle=\color{ForestGreen}
}

\setcounter{section}{0}

\begin{document}

\maketitle

\section{Introduction}

The problem is composed in two parts:

\begin{enumerate}
    \item Consider the Bloom filters where a single random universal hash random function $h : U \to [m]$ is employed for a set $S \subseteq U$ of keys, where $U$ is the universe of keys.
    Consider its binary array $B$ of $m$ bits. Suppose that $m \ge c|S|$, for some constant $c > 1$, and that both $c$ and $|S|$ are unknown to us.   
    Estimate the expected number of 1s in $B$ under a uniform choice at random of $h \in H$. Is this related to $|S|$? Can we use it to estimate $|S|$? 
    
    \item Consider $B$ and its rank function: show how to use extra $O(m)$ bits to store a space-efficient data structure that returns, for any given $i$, the following answer in constant time: 
    $\textrm{rank}(i) = \#1s \in B[1..i]$
    
    \textit{Hint}: Easy to solve in extra O(m log m) bits. To get O(m) bits, use prefix sums on B, and sample them. Use a lookup table for pieces of B between any two consecutive samples. 
\end{enumerate}

\section{Solution}
TODO

\end{document}
